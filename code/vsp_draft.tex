\documentclass[12pt]{article}

% ----------- PACKAGES -----------
\usepackage{booktabs}     % professional-quality tables
\usepackage{caption}      % better caption formatting
\usepackage{geometry}     % adjust page margins
\usepackage{lmodern}      % clean font
\usepackage{setspace}     % for line spacing
\usepackage{threeparttable} % optional notes below table
\usepackage{natbib}
\usepackage{comment}
\usepackage{graphicx}
\usepackage{titling}
\usepackage{multirow}
\usepackage{makecell}
\usepackage{booktabs}

% ----------- PAGE SETUP -----------
\geometry{margin=1in}
\setstretch{1.1}

\title{Does Foreign Aid Improve Education Outcomes? Evidence from Uganda \\~\\ \Large VSP Draft 1}
\author{Riana Ramonjamanana \and Charlene R. Ramos}
\date{2025-11-13}

\begin{document}
\maketitle
% The VSP should include:
% a) an introduction, which generates interest in the exercise,
% b) a derivation or motivation for the estimating equation,
% c) a table of means and as many tables of estimation results as you like,
% d) a written discussion of the table of means and of estimation results, and
% e) a conclusion

\section{Introduction}

In this paper, we study the impact of foreign aid on the education outcomes of Ugandans. The relationship between the amount of aid received and the economic growth of developing countries is subject to debate. Using spatial data, we trace the geographic boundaries of projects and their reach during 2014 to 2017. Our analysis attempts to address the void between cross-country studies and micro-level impact evaluations by employing the subnational approach. This lens offers greater external and ecological validity over individual trials. 

The existing literature is divided on whether aid augments or harms education outcomes. Researchers \citet{BazziClemens2013} posit that aid acts as a buffer to stabilize fiscal policy and guards households against adverse shocks. The authors shed light on how aid sustains education spending and thereby supports children living in low-income countries to stay in school. \citet{Duflo2001} adds that expanded government budget enables teacher recruitment and accelerates school construction. Her landmark study in Indonesia provides evidence that partially aid-financed efforts raised years of schooling and enrollment rates among disadvantaged children. 

Aid can also be reframed as misallocation or regarded as poor budget discipline \citep{VandeSijpe2013}. For example, \citet{CelasunWalliser2008} asserts that fiscal dependence on donor financing distorts general equilibrium effects because governments divert domestic funds away from education in response to overcrowding in public investment. Such unintended consequences impose transaction costs due to weakened coordination. These may perpetuate systemic inequities and impede social mobility for impoverished populations \citep{Gehring2017}. 

The mixed evidence underscores the need for subnational analyses that can capture the heterogeneous and context-specific effects of aid on education. Our study contributes to this literature by leveraging geocoded aid data and district-level education outcomes to examine these dynamics within Uganda. The remainder of the paper proceeds as follows: Section~\ref{sec:data} describes the data sources utilized. Section~\ref{sec:methodology} outlines our empirical approach. Section~\ref{sec:result} presents the estimation results. Section~\ref{sec:discuss} discusses our findings and draws conclusions.
%aid doesn't work everywhere, explore non linearity \cite{BazziClemens2013}
%whether aid projects are effective on average, which continues to be subject to debate in the literature. \cite{heinzel2024trust}

\section{Data}
\label{sec:data}

\subsection{\textbf{Variables}}
\begin{itemize}
    \item \textbf{Education.} Ministry of Education \& Sports publishes annual reports to monitor progress in achieving equitable access to  quality education in Uganda. Our analysis leveraged primary and secondary gross enrollment rates by district. Gross enrollment rate (GER) in Table~\ref{tab:summary} is defined as the ratio of the total number of students enrolled at a given level of education to the age population corresponding to that level.\footnote{Values greater than 100\% indicate that over-age and under-age students are enrolled.}
    \item \textbf{Foreign Aid.} The transactions detailed in the foreign aid dataset was extracted using the GeoQuery platform by development research lab AidData. The collection contains 565 geocoded projects across 2,426 locations dated from the late 1970s through the mid-2010s. In total, the database accounts for over \$$7.5$ billion in disbursements. Table~\ref{tab:summary} reports aid disbursements, which is our main explanatory variable. Disbursement is assigned to districts by evenly splitting each project’s total amount across its recipients.
\end{itemize}


% ----------- TABLE 1-----------
\setlength{\tabcolsep}{12pt}  % default is 6pt
\begin{table}[htb]
\centering
\begin{threeparttable}
\caption{Summary Statistics by Year}
\label{tab:summary}
\small
\begin{tabular}{l c c c c c}
\toprule
 & N & 2014 & 2015 & 2016 & 2017 \\
\midrule
\multicolumn{6}{l}{\textbf{Education Outcomes}}\\
Primary gross enrollment rate & 458
    & \makecell[c]{125.16 \\ (43.53)}
    & \makecell[c]{111.12 \\ (27.03)}
    & \makecell[c]{128.96 \\ (66.90)}
    & \makecell[c]{115.48 \\ (30.45)} \\[0.3cm]
\addlinespace
Secondary gross enrollment rate & 458
    & \makecell[c]{29.62 \\ (15.96)}
    & \makecell[c]{23.60 \\ (13.55)}
    & \makecell[c]{28.65 \\ (19.28)}
    & \makecell[c]{24.29 \\ (12.47)} \\[0.3cm]
\multicolumn{6}{l}{\textbf{Aid and Covariates}}\\
Aid & 153
    & \makecell[c]{17.73 \\ (2.12)}
    & \makecell[c]{15.64 \\ (2.23)}
    & \makecell[c]{14.47 \\ (1.53)}
    & \makecell[c]{16.25 \\ (2.61)} \\[0.3cm]
\addlinespace
Nighttime lights & 458
    & \makecell[c]{0.098 \\ (0.83)}
    & \makecell[c]{0.097 \\ (0.82)}
    & \makecell[c]{0.093 \\ (0.78)}
    & \makecell[c]{0.103 \\ (0.86)} \\[0.3cm]
\addlinespace
Population density & 458
    & \makecell[c]{12.33 \\ (0.58)}
    & \makecell[c]{12.37 \\ (0.58)}
    & \makecell[c]{12.40 \\ (0.58)}
    & \makecell[c]{12.41 \\ (0.57)} \\[0.3cm]
\addlinespace
Rainfall & 458
    & \makecell[c]{6.24 \\ (0.73)}
    & \makecell[c]{6.34 \\ (0.75)}
    & \makecell[c]{6.22 \\ (0.75)}
    & \makecell[c]{6.29 \\ (0.76)} \\
\bottomrule
\end{tabular}
\begin{tablenotes}[flushleft]\small
\item Notes: Values are means with standard deviations in parentheses. Aid, population density, and rainfall are expressed in log form. Aid data are from Uganda's Aid Management Platform that compiles disbursements for every geocoded project through 2014. Even-split disbursement represent evenly allocated total disbursement amounts between project start and end years. Education data are from the Uganda Bureau of Statistics.
\end{tablenotes}
\end{threeparttable}

\end{table}

\subsection{\textbf{Controls}}
\begin{itemize}
    \item \textbf{Nighttime lights (NTL).} Nighttime lights from AidData proxy local economic activity and infrastructure. We include the annual average VIIRS nighttime lights value in the control vector $X_{it}$ to capture changes in development that could confound the aid–education relationship. As shown in Figure~\ref{fig:ntl_uganda}, their spatial distribution closely mirrors the urban–rural divide.
    \item \textbf{Population density.} District-level population is measured using WorldPop’s 1km-resolution population estimates. These data provide high-resolution counts of the number of people residing in each district. Incorporating population adjusts for differences in demographic scale that affect aid allocation and education outcomes.
    %create hyperlink to data source later
    \item \textbf{Rainfall.} Annual precipitation totals from the University of Delaware’s UDel dataset give yearly rainfall in millimeters. Rainfall is the key source of weather variation in Uganda, as the country’s equatorial position leads to minimal temperature changes both within and across years \citep{BjorkmanNyqvist2013}.
\end{itemize}

% ----------- FIGURE 1-----------
\begin{figure}[htb]
    \centering
    \includegraphics[width=0.85\textwidth]{fig_ntl_uganda.png}
    \caption{Nighttime Lights Map of Uganda, 2017.}
    \label{fig:ntl_uganda}
\end{figure}





\subsection{\textbf{Limitations}}
There are numerous challenges when conducting a subnational approach related to data accessibility and quality \citep{Nunnenkamp2016}. First, it is widely acknowledged by researchers that regional aid data is sparse and only available for a few countries. Incomplete, fragmented, and inconsistent reporting by donors is prolific, which complicates construction of a complete, accurate time series. Second, the exactness of geocoding is uneven among projects and renders several observations unsuitable for analysis. Such observations are imprecise because they correspond to the entire nation, an administrative region, or estimated coordinates based on landscape features (like rivers). Third, the AidData database was last updated in April 2016 and still only accounts for projects with start dates from 1978 through 2014. The effective start date is also missing for more than a tenth of projects. In terms of education data, the lag in publication, poor record keeping by education institutions, low response rate of private schools, and misreporting of student information may compromise the integrity of data collection efforts. 

% Raj de, maalwi, section 4.data

\section{Methodology}
\label{sec:methodology}
We aim to estimate the causal effect of aid on education. Several identification challenges arise when estimating the relationship between education outcomes and aid using simple regression techniques. First, reverse causality may bias the estimates: districts with low education outcomes may receive more aid. Second, simultaneity may occur if unobserved district attributes simultaneously depress education and attract higher aid allocations. Third, aid disbursement is measured imprecisely. The data do not capture the exact amount received by each district; project-level disbursements are often evenly divided across multiple beneficiary districts, introducing noise into the measure of aid exposure.

A range of approaches has been proposed in the literature to address these concerns. In this study, we employ two complementary strategies.%: an ordinary least squares regression that accounts for the aforementioned issues and an instrumental variable approach.

First, we follow \citet{BazziClemens2013} and adopt an empirical strategy structured to mitigate key endogeneity concerns. Since contemporaneous aid may correlate with contemporaneous education outcomes, we use lagged aid to allow for delayed effects of aid on education. We then take the first difference of the outcome variable to eliminate the influence of time-invariant unobserved district characteristics. Ideally, we would further restrict aid to education-focused and early-impact categories to isolate channels most likely to affect enrollment. However, due to data limitations—and because our objective is to estimate a general equilibrium effect rather than the impact of specific education interventions—we retain all aid received by each district. Taken together, these steps constitute our preferred ordinary least squares specification:
\begin{equation}
   \Delta\text{Edu}_{it}=\alpha + \beta \text{Aid}_{it-1} + \text{X}_{it} \gamma +\epsilon_{it}
\end{equation}
where \( i \) and \( t \) index district and year, respectively. $\Delta\text{Edu}_{it}$ denotes the yearly change in the primary or secondary gross enrollment rate, while \( \text{Aid}_{it-1} \) is lagged aid disbursement. \( \text{X}_{it} \) is a vector of covariates including rainfall, population, and nighttime lights.

Aid and education are jointly determined by a wide set of factors that are difficult to fully account for. To further address potential endogeneity, we therefore employ an instrumental variable approach, widely used in the aid literature, though no consensus exists on the most credible instrument.

We exploit exogenous variation in aid arising from differences in donor–government dynamics using a Bartik (shift–share) instrumental variable design. We assume that increases in government expenditure in donor countries translate into larger aid budgets. Our instrument for bilateral aid is the interaction between donor government expenditure as a share of GDP in each period and a time-invariant district-level probability of receiving aid from each donor, \(p_{i,j}\). Specifically, \(p_{i,j}\) measures the percentage of pre-period years (2000–2013) in which district \(i\) received aid from donor \(j\).

The instrument is defined as:
\[
ss_{i,t} = \sum_{j} \left( \frac{\text{Government \, expenditure}}{GDP} \right)_{j,t} \times p_{i,j}
\]

where the exposure share $p_{i,j}$ for district $i$ and donor $j$ is:
\[
p_{i,j} = \frac{1}{T} \sum_{t=1}^{T} p_{i,j,t},
\]
with $p_{i,j,t}$ indicating whether district $i$ received a positive amount of aid from donor $j$ in year $t$.

The strength of this instrument is evaluated using the first-stage regression. We assume that donor government expenditure is unrelated to education outcomes in the recipient country, ensuring the excludability of the instrument.



%not excludable.

\begin{comment}
We also explore decreasing returns to aid by including aid squared.

 fractionalization on aid for
regular and irregular recipient countries.

%avoiding reliance on potentially weak or invalid instruments while


we will instrument district-level aid receipts with the interaction of fractionalization with the share of years in our sample that a district receives aid from its donors. "Volkerink and de Haan (2001) and Scartascini and Crain (2002) show that legislature fragmentation increases governments’ expenditures"


other instrument" Population size but not excludable, Political-relations based variables reflects the effects of politically motivated aid rather than those of all aid, 

To the extent that variables correlated with donor fractionalization do not affect recipients’ rates of growth differently in regular and irregular recipients of aid, controlled for country and period fixed effects, the resulting instrument is excludable. Contrary to Nunn and Qian (2014) and Ahmed (2016), we focus on growth rather than democracy or conflict and on aid from a group of major donors rather than (food) aid from the United States exclusively. Other than Werker et al. (2009), we focus on a broad set of donor countries. 

expenditures as a share of GDP significantly determine aid budgets.


Potential issues: \\
reverse causality aid<-> education, selection: region with bad outcome receives more aid, endogeneity:  unobserved variables related to both aid and education, that make the role of aid appear significant\\
simultaneity
\\
instrument must be relevantcov(X,Z)<>0 and exogenous cov(z,e)=0



Possible instrument
\begin{itemize}
    \item Ethnic affinity as instrument:  the proportion of the population in a district or constituency that is coethnic with the sitting President
    \item Political switching
\end{itemize}

Relevance was tested through standard first stage regressions and exclusion was based on some obvious theoretical knowledge{such as not using a diarrhea variable to measure the change in diarrhea incidence{and passing the overidentification test.
\end{comment}

\section{Results}
\label{sec:result}

\subsection{\textbf{OLS Regression Estimation}}

Tables~\ref{tab:ols_primary} and \ref{tab:ols_second} report ordinary least squares (OLS) estimates for primary and secondary gross enrollment rates (GERs). Columns (1)--(2) use the levels of GER as the dependent variable, while columns (3)--(4) utilize its first difference. Aid, population, and rainfall are expressed in logarithms. Nighttime lights enter in levels.

\begin{itemize}
    \item \textbf{Primary Gross Enrollment Rate (GER).} Across specifications, there exists no statistically significant relationship between contemporaneous aid and primary GER. In column (4), the lagged-aid coefficient is positive but small and imprecise, suggesting at most a weak link between prior-year aid and short-run changes in primary GER. Nighttime lights and rainfall display consistently negative correlations with primary GER, capturing persistent spatial disparities in infrastructure and climate.

% ----------- TABLE 2-----------
\setlength{\tabcolsep}{12pt}  % default is 6pt
\begin{table}[htb]
\centering
\begin{threeparttable}
\caption{OLS Estimation for Primary GER}
\label{tab:ols_primary}
\small
\begin{tabular}{lcccc}
\toprule
 & (1) & (2) & (3) & (4) \\
 & Level & Level & First-difference & First-difference \\
\midrule
Aid & $-1.79$ &  & $-3.55^{*}$ &  \\
 & (1.16) &  & (1.97) &  \\
\addlinespace
Aid (lag)$ &  & $-1.50$ &  & 0.55 \\
 &  & (1.02) &  & (1.09) \\
\addlinespace
Nighttime lights & $-7.55^{**}$ & $-6.67^{***}$ & $-0.08$ & 2.34 \\
 & (3.04) & (2.34) & (2.06) & (2.94) \\
\addlinespace
Population density & 2.37 & 4.26 & $-0.97$ & $-2.07$ \\
 & (9.85) & (7.65) & (4.97) & (7.77) \\
\addlinespace
Rainfall & $-13.42^{*}$ & $-13.16^{**}$ & $-0.90$ & $-2.16$ \\
 & (7.06) & (5.27) & (4.94) & (7.14) \\
\addlinespace
Constant & $211.05^{**}$ & $172.34^{*}$ & 75.38 & 20.08 \\
 & (103.99) & (86.85) & (77.69) & (69.94) \\
\addlinespace
\midrule
Observations & 153 & 126 & 68 & 126 \\
R-squared & 0.06 & 0.09 & 0.08 & 0.01 \\
\bottomrule
\end{tabular}
\begin{tablenotes}[flushleft]\small
\item Notes: Robust standard errors in parentheses, clustered by district. 
Aid, population density, and rainfall expressed in log form.
In columns (1)–(2),  dependent variable is in levels; 
columns (3)–(4) use its first difference.
*, **, and *** denote significance at the 10\%, 5\%, and 1\% levels, respectively.
\end{tablenotes}
\end{threeparttable}
\end{table}

    \item \textbf{Secondary Gross Enrollment Rate (GER).} Secondary GER shows a similar pattern. Aid remains statistically insignificant across specifications, with point estimates near zero and sensitive to lag structure. The negative association between nighttime lights and GER arises from underlying demographic trends: highly urbanized and well-lit districts demonstrate age-appropriate grade progression with lower repetition rates in school.\footnote{Urban districts have higher NTL but lower GER because older repeaters are fewer, repetition is lower, and youth progress faster through grades.} Moreover, districts characterized by heavier rainfall tend to be rural, agricultural, less accessible, and more climate vulnerable. Rainfall physically disrupts schooling and thus reduces GER. 

% ----------- TABLE 3-----------
\begin{table}[htb]
\centering
\begin{threeparttable}
\caption{OLS Estimation for Secondary GER}
\label{tab:ols_second}
\small
\begin{tabular}{lcccc}
\toprule
 & (1) & (2) & (3) & (4) \\
 & Level & Level & First-difference & First-difference \\
\midrule
Aid & 0.07 &  & $-0.57$ &  \\
 & (0.30) &  & (0.45) &  \\
\addlinespace
Aid (lag) &  & $-0.33$ &  & $-0.42$ \\
 &  & (0.37) &  & (0.28) \\
\addlinespace
Nighttime lights & $-3.42^{***}$ & $-2.22^{***}$ & $-0.19$ & $2.24^{***}$ \\
 & (0.75) & (0.66) & (0.59) & (0.51) \\
\addlinespace
Population density & $7.69^{***}$ & $6.75^{***}$ & $-0.11$ & $-2.58$ \\
 & (2.43) & (2.56) & (1.47) & (1.62) \\
\addlinespace
Rainfall & $-11.30^{***}$ & $-10.48^{***}$ & 1.77 & 0.62 \\
 & (2.62) & (2.53) & (1.65) & (1.07) \\
\addlinespace
Constant & 2.36 & 13.00 & $-2.61$ & $31.16^{*}$ \\
 & (37.48) & (42.78) & (24.71) & (17.21) \\
\addlinespace
\midrule
Observations & 153 & 126 & 68 & 126 \\
R-squared & 0.27 & 0.28 & 0.06 & 0.07 \\
\bottomrule
\end{tabular}
\begin{tablenotes}[flushleft]\small
\item Notes: Robust standard errors in parentheses, clustered by district. 
Aid, population density, and rainfall expressed in log form.
In columns (1)–(2),  dependent variable is in levels; 
columns (3)–(4) use its first difference.
*, **, and *** denote significance at the 10\%, 5\%, and 1\% levels, respectively.
\end{tablenotes}
\end{threeparttable}
\end{table}

\end{itemize}

\subsection{\textbf{Instrumental Variable (IV) Analysis}}
To address potential endogeneity in aid allocation, we implement an instrumental variable (IV) specification. This involves employing a shift-share instrument based on historical district exposure to major bilateral donors interacted with annual changes in national disbursements. This design isolates plausibly exogenous variation in aid flows driven by external budget cycles.

Column (1) of Table~\ref{tab:iv_ger} presents a strong first stage: the instrument significantly predicts district-level aid, with a coefficient of 2.54. Also, the implied first-stage F-statistic exceeds conventional thresholds and provides evidence of instrument relevance. Columns (2)--(3) show reduced-form estimates of donor shocks on primary and secondary GER. These coefficients are small and tightly centered around zero, indicating no detectable intent-to-treat effect of donor-driven aid shifts on enrollment.

Columns (4)--(5) report the second-stage IV estimates, which recover the causal effect of exogenous aid variation on GER. Consistent with the OLS and reduced-form results, the IV coefficients are close to zero and statistically insignificant. The estimates suggest that, within the short 2014--2017 window, donor-driven variation in district aid does not produce measurable changes in enrollment outcomes at either the primary or secondary level.

Although nighttime lights and rainfall are negatively correlated with enrollment, these relationships do not threaten the validity of the IV estimation strategy. The shift-share instrument draws upon exogenous donor disbursement shocks interacted with baseline district exposure, thereby relying on variation originating outside the education system. The negative correlations in the OLS regressions reflect geographic and socioeconomic heterogeneity––urban districts with lower repetition, or high-rainfall agricultural districts with limited school accessibility–—rather than any systematic channel linking these controls to the instrument. Since donor budget shocks are determined externally and cannot be influenced by district-level characteristics, nighttime lights and rainfall do not bias the second-stage IV estimates. Instead, they underscore the importance of controlling for spatial and climatic differences in the OLS models, while leaving the IV results robust to potential confounding.

% ----------- TABLE 4-----------
\setlength{\tabcolsep}{12pt}  % default is 6pt
\begin{table}[htb]
\centering
\begin{threeparttable}
\caption{IV Estimation for GERs}
\label{tab:iv_ger}
\small
\begin{tabular}{lccccc}
\toprule
 & (1) & (2) & (3) & (4) & (5) \\
 & \multirow{2}{*}{FS} & RF  & IV  & RF  & IV \\
 &  & Primary & Primary & Secondary & Secondary \\

\midrule
Aid &  &  & $0.00$   &  & $-0.01$  \\
    &  &  & $(2.00)$ &  & $(0.58)$ \\
\addlinespace
Instrument & $2.54^{***}$   & $0.00$   &  & $-0.02$  &  \\
           & $(0.26)$       & $(4.50)$ &  & $(1.30)$ &  \\
\addlinespace
Nighttime lights   & $-1.59^{***}$ & $-1.19$   & $-1.19$   & $-0.04$   & $-0.04$  \\
                   & (0.36)        & $(5.33)$  & $(4.97)$  & $(1.53)$  & $(1.43)$ \\
\addlinespace
Population density & $-0.20$   & $-1.55$   & $-1.55$   & $-0.37$   & $-0.39$  \\
                   & $(1.05)$  & $(5.55)$  & $(6.42)$  & $(1.72)$  & $(1.82)$ \\
\addlinespace
Rainfall & $-1.26^{**}$ & $-2.98$  & $-2.98$  & $1.54$   & $1.56$   \\
         & $(0.59)$     & $(7.26)$ & $(9.27)$ & $(2.30)$ & $(2.87)$ \\
\addlinespace
Constant & $24.01^{*}$  & $41.05$   & $41.05$   & $-6.57$   & $-6.33$   \\
         & (12.98)      & $(83.43)$ & $(81.60)$ & $(27.42)$ & $(24.51)$ \\
\addlinespace
\midrule
Observations & $144$ & $63$ & $63$ & $63$ & $63$ \\
R-squared & $0.44$ & $0.00$ & $0.00$ & $0.02$ & $0.02$ \\
\bottomrule
\end{tabular}
\begin{tablenotes}[flushleft]\small
\item Notes: Robust standard errors in parentheses, clustered by district. 
FS = first stage; RF = reduced form; IV = instrumental variable.
An observation combines district and year. 
Population density and rainfall expressed in log form.
Sample restricted to bilateral aid from EU, Japan, Norway, UK, and USA (2014--2017).
*, **, and *** denote significance at the 10\%, 5\%, and 1\% levels, respectively.
\end{tablenotes}
\end{threeparttable}
\end{table}



% Short time frame the most detectable effects will be outcomes that can improve shortly after aid is disbursed.



\section{Discussion and Conclusion}
\label{sec:discuss}
This study investigates the causal effect of aid on education using district-level data from Uganda during 2014––2017. Our estimates reveal no effect on the short-run change in GERs, consistent with earlier evidence in the literature \citep{Nunnenkamp2016}. Several factors may account for these findings. First, aid may not instantly affect outcomes. GER is a slow-moving, aggregate indicator and less sensitive to yearly fluctuations in aid budgets. Moreover, aid often finances infrastructure, capacity building, and administrative systems whose effects materialize gradually rather than producing immediate shifts in enrollment. Second, our estimates may be biased downward if donors increase aid in response to political crises or other adverse shocks expected to weaken education performance. In such cases, aid systematically flows to districts already anticipating declines in enrollment, generating a negative correlation that understates the true effect of aid. Third, the data span only four years, while the number of districts reaches 122 in the full sample. This short time dimension limits our ability to control for time-invariant district characteristics or to capture richer dynamic patterns. Finally, measurement error may obscure the results. Aid is not perfectly recorded; in addition, projects initiated after 2014 are not accounted for. This introduces noise that attenuate estimated effects. Due to these limitations, we cannot claim causality. Expanding the dataset would increase statistical power, allowing us to exploit heterogeneity. Although our results suggest that donor-driven variations in district aid flows do not translate into prompt changes in schooling participation, future research might explore other dimensions of educational quality.

\newpage

% ----------- References -----------
\bibliographystyle{chicago}
\bibliography{references}

\end{document}


