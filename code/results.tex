\documentclass[12pt]{article}

% ----------- PACKAGES -----------
\usepackage{booktabs}     % professional-quality tables
\usepackage{caption}      % better caption formatting
\usepackage{geometry}     % adjust page margins
\usepackage{lmodern}      % clean font
\usepackage{setspace}     % for line spacing
\usepackage{threeparttable} % optional notes below table
\usepackage{natbib}
% \usepackage{comment}
\usepackage{graphicx}

% ----------- PAGE SETUP -----------
\geometry{margin=1in}
\setstretch{1.1}

\begin{document}

\section{Results}
\label{sec:result}

\subsection{\textbf{OLS Regression Estimation}}

\begin{itemize}
    \item Primary Gross Enrollment Rate (GER)




% ----------- TABLE 2-----------
\setlength{\tabcolsep}{12pt}  % default is 6pt
\begin{table}[htbp]
\centering
\begin{threeparttable}
\caption{OLS Estimation for Primary GER}
\label{tab:ols_primary}
\begin{tabular}{lcccc}
\toprule
 & (1) & (2) & (3) & (4) \\
 & Level & Level & Growth Rate & Growth Rate \\
\midrule
Aid & $-1.79$ &  & $-3.55^{*}$ &  \\
 & (1.16) &  & (1.97) &  \\
\addlinespace
Aid (lag)$ &  & $-1.50$ &  & 0.55 \\
 &  & (1.02) &  & (1.09) \\
\addlinespace
Nighttime lights & $-7.55^{**}$ & $-6.67^{***}$ & $-0.08$ & 2.34 \\
 & (3.04) & (2.34) & (2.06) & (2.94) \\
\addlinespace
Population density & 2.37 & 4.26 & $-0.97$ & $-2.07$ \\
 & (9.85) & (7.65) & (4.97) & (7.77) \\
\addlinespace
Rainfall & $-13.42^{*}$ & $-13.16^{**}$ & $-0.90$ & $-2.16$ \\
 & (7.06) & (5.27) & (4.94) & (7.14) \\
\addlinespace
Constant & $211.05^{**}$ & $172.34^{*}$ & 75.38 & 20.08 \\
 & (103.99) & (86.85) & (77.69) & (69.94) \\
\addlinespace
\midrule
Observations & 153 & 126 & 68 & 126 \\
R-squared & 0.06 & 0.09 & 0.08 & 0.01 \\
\bottomrule
\end{tabular}
\begin{tablenotes}[flushleft]\small
\item Notes: Robust standard errors in parentheses, clustered by district. 
Aid, population density, and rainfall expressed in log form.
In columns (1)–(2),  dependent variable is level of education; 
columns (3)–(4) use its first difference.
*, **, and *** denote significance at the 10\%, 5\%, and 1\% levels, respectively.
\end{tablenotes}
\end{threeparttable}
\end{table}


    \item Secondary Gross Enrollment Rate (GER)

% ----------- TABLE 3-----------
\begin{table}[htbp]
\centering
\begin{threeparttable}
\caption{OLS Estimation for Secondary GER}
\label{tab:ols_second}
\begin{tabular}{lcccc}
\toprule
 & (1) & (2) & (3) & (4) \\
 & Level & Level & Growth Rate & Growth Rate \\
\midrule
Aid & 0.07 &  & $-0.57$ &  \\
 & (0.30) &  & (0.45) &  \\
\addlinespace
Aid (lag) &  & $-0.33$ &  & $-0.42$ \\
 &  & (0.37) &  & (0.28) \\
\addlinespace
Nighttime lights & $-3.42^{***}$ & $-2.22^{***}$ & $-0.19$ & $2.24^{***}$ \\
 & (0.75) & (0.66) & (0.59) & (0.51) \\
\addlinespace
Population density & $7.69^{***}$ & $6.75^{***}$ & $-0.11$ & $-2.58$ \\
 & (2.43) & (2.56) & (1.47) & (1.62) \\
\addlinespace
Rainfall & $-11.30^{***}$ & $-10.48^{***}$ & 1.77 & 0.62 \\
 & (2.62) & (2.53) & (1.65) & (1.07) \\
\addlinespace
Constant & 2.36 & 13.00 & $-2.61$ & $31.16^{*}$ \\
 & (37.48) & (42.78) & (24.71) & (17.21) \\
\addlinespace
\midrule
Observations & 153 & 126 & 68 & 126 \\
R-squared & 0.27 & 0.28 & 0.06 & 0.07 \\
\bottomrule
\end{tabular}
\begin{tablenotes}[flushleft]\small
\item Notes: Robust standard errors in parentheses, clustered by district. 
Aid, population density, and rainfall expressed in log form.
In columns (1)–(2),  dependent variable is level of education; 
columns (3)–(4) use its first difference.
*, **, and *** denote significance at the 10\%, 5\%, and 1\% levels, respectively.
\end{tablenotes}
\end{threeparttable}
\end{table}




\end{itemize}

\subsection{\textbf{Instrumental Variable (IV) Analysis}}

% ----------- TABLE 4-----------
\begin{table}[htbp]
\centering
\begin{threeparttable}
\caption{IV Estimation for GERs}
\label{tab:iv_ger}
\begin{tabular}{lccccc}
\toprule
 & (1) & (2) & (3) & (4) & (5) \\
 & FS & RF Primary & IV Primary & RF Secondary & IV Secondary \\
\midrule
Aid &  &  & $0.00$   &  & $-0.01$  \\
    &  &  & $(2.00)$ &  & $(0.58)$ \\
\addlinespace
Instrument & $2.54^{***}$   & $0.00$   &  & $-0.02$  &  \\
           & $(0.26)$       & $(4.50)$ &  & $(1.30)$ &  \\
\addlinespace
Nighttime lights   & $-1.59^{***}$ & $-1.19$   & $-1.19$   & $-0.04$   & $-0.04$  \\
                   & (0.36)        & $(5.33)$  & $(4.97)$  & $(1.53)$  & $(1.43)$ \\
\addlinespace
Population density & $-0.20$   & $-1.55$   & $-1.55$   & $-0.37$   & $-0.39$  \\
                   & $(1.05)$  & $(5.55)$  & $(6.42)$  & $(1.72)$  & $(1.82)$ \\
\addlinespace
Rainfall & $-1.26^{**}$ & $-2.98$  & $-2.98$  & $1.54$   & $1.56$   \\
         & $(0.59)$     & $(7.26)$ & $(9.27)$ & $(2.30)$ & $(2.87)$ \\
\addlinespace
Constant & $24.01^{*}$  & $41.05$   & $41.05$   & $-6.57$   & $-6.33$   \\
         & (12.98)      & $(83.43)$ & $(81.60)$ & $(27.42)$ & $(24.51)$ \\
\addlinespace
\midrule
Observations & $144$ & $63$ & $63$ & $63$ & $63$ \\
R-squared & $0.44$ & $0.00$ & $0.00$ & $0.02$ & $0.02$ \\
\bottomrule
\end{tabular}
\begin{tablenotes}[flushleft]\small
\item Notes: Robust standard errors in parentheses, clustered by district. 
FS = first stage; RF = reduced form; IV = instrumental variable.
An observation combines district and year. 
Population density and rainfall expressed in log form.
Sample restricted to bilateral aid from EU, Japan, Norway, UK, and USA (2014--2017).
*, **, and *** denote significance at the 10\%, 5\%, and 1\% levels, respectively.
\end{tablenotes}
\end{threeparttable}
\end{table}




\section{Results}

\subsection{OLS Estimates}

Tables~\ref{tab:ols_primary} and \ref{tab:ols_secondary} report OLS estimates for primary and secondary gross enrollment rates (GER). Columns (1)--(2) use the level of the education outcome as the dependent variable, while columns (3)--(4) use its first difference. Aid, population, and rainfall are expressed in natural logarithms; nighttime lights enter in levels.

Across all specifications, we find no statistically significant relationship between contemporaneous aid and primary GER. In column (4), the lagged-aid coefficient is positive but small and imprecisely estimated, suggesting at most a weak link between prior-year aid and short-run changes in primary enrollment. Nighttime lights and rainfall display consistently negative correlations with primary GER levels, capturing persistent spatial disparities in infrastructure, accessibility, and climatic conditions.

Secondary GER shows a similar pattern. Aid remains statistically insignificant across specifications, with point estimates close to zero and sensitive to lag structure. The negative association between nighttime lights and GER reflects a mechanical feature of the Ugandan setting: highly urbanized and well-lit districts tend to have age-appropriate grade progression and lower repetition rates, and therefore exhibit lower \emph{gross} enrollment ratios despite higher socioeconomic development.

\null\vspace{1em}

Although nighttime lights and rainfall display strong negative correlations with enrollment, these relationships do not threaten the validity of the instrumental-variables strategy. The shift--share instrument draws upon exogenous donor disbursement shocks interacted with baseline district exposure and therefore relies on variation originating outside the education system. The negative correlations in the OLS regressions reflect geographic and socioeconomic heterogeneity—urban districts with lower repetition, or high-rainfall agricultural districts with limited school accessibility—rather than any systematic channel linking these controls to the instrument. Because donor budget shocks are determined externally and cannot be influenced by district-level characteristics, these background correlations do not bias the second-stage IV estimates. Instead, they underscore the importance of controlling for spatial and climatic differences in the OLS specifications, while leaving the IV results robust to potential confounding.

\vspace{1em}

\subsection{Instrumental-Variables Estimates}

To address potential endogeneity in aid allocation, we employ an instrumental-variables approach using a shift--share instrument based on historical district exposure to major bilateral donors interacted with annual changes in national donor disbursements. This design isolates plausibly exogenous variation in aid flows driven by external donor budget cycles.

Column (1) of Table~\ref{tab:iv_ger} presents a strong first stage: the instrument significantly predicts district-level aid, with a coefficient of 2.54 (s.e.\ 0.26) and an implied first-stage F-statistic far exceeding conventional thresholds. This provides clear evidence of instrument relevance. Columns (2)--(3) show reduced-form estimates of donor shocks on primary and secondary GER. These coefficients are small, precisely estimated around zero, and statistically insignificant, indicating no detectable intent-to-treat effect of donor-driven aid shifts on enrollment.

Columns (4)--(5) report the second-stage IV estimates, which recover the causal effect of exogenous aid variation on GER. Consistent with the OLS and reduced-form results, the IV coefficients are close to zero and statistically insignificant. The estimates suggest that, within the short 2014--2017 window, donor-driven variation in district aid does not produce measurable changes in enrollment outcomes at either the primary or secondary level.

\subsection{Interpretation}

Taken together, the OLS and IV evidence indicates that district-level aid does not generate short-run changes in gross enrollment rates in Uganda during 2014--2017. The absence of detectable impacts is consistent with the slow-moving nature of GER as an aggregate indicator, the limited sensitivity of enrollment to year-to-year budget fluctuations, and potential measurement error in administrative education data. Aid often finances infrastructure, capacity building, and administrative systems whose effects materialize gradually and may not immediately shift enrollment patterns. The short panel further limits statistical power to detect dynamic responses. Overall, the findings suggest that donor-driven variations in district aid flows do not translate into immediate changes in schooling participation, although they may influence other dimensions of educational quality or longer-term outcomes not captured by GER.





% Short time frame the most detectable effects will be outcomes that can improve shortly after aid is disbursed.

% ----------- TABLE -----------
% \begin{table}[h]
% \centering
% \caption{Effect of Aid on Education Outcomes}
% \label{tab:aid_linear}
% \begin{threeparttable}
% \begin{tabular}{lcc}
% \toprule
%                                  & (1) Primary GER    & (2) Secondary GER    \\
% \midrule
% Disbursements                   & 0.58     & -0.37     \\
%                                 & (1.14)   & (0.29)    \\
% \addlinespace
% Nighttime lights                & 65.91    & 13.60     \\
%                                 & (55.16)  & (16.05)   \\
% \addlinespace
% Population density              & -2.36    & 1.17      \\
%                                 & (13.06)  & (3.22)    \\
% \addlinespace
% Rainfall                        & -1.68    & -0.05     \\
%                                 & (7.99)   & (1.33)    \\
% \addlinespace
% Government spending            & -2.08     & -4.90*    \\
%                                & (10.65)   & (2.86)    \\
% \addlinespace
% Constant                       & 66.56     & 100.80**  \\
%                                & (176.01)  & (43.57)   \\
% \midrule
% Observations                   & 121       & 121       \\
% $R^{2}$                        & 0.01      & 0.07      \\
% \bottomrule
% \end{tabular}
% \begin{tablenotes}
% \footnotesize \item Notes: The dependent variables are the primary and secondary gross enrollment rates. All explanatory variables are expressed in log except for the nighttime lights. Robust standard errors in parentheses. 
% *, **, and *** denote significance at the 10\%, 5\%, and 1\% levels. 
% \end{tablenotes}
% \end{threeparttable}
% \end{table}

\end{document}


