% Options for packages loaded elsewhere
\PassOptionsToPackage{unicode}{hyperref}
\PassOptionsToPackage{hyphens}{url}
\documentclass[
]{article}
\usepackage{xcolor}
\usepackage[margin=1in]{geometry}
\usepackage{amsmath,amssymb}
\setcounter{secnumdepth}{5}
\usepackage{iftex}
\ifPDFTeX
  \usepackage[T1]{fontenc}
  \usepackage[utf8]{inputenc}
  \usepackage{textcomp} % provide euro and other symbols
\else % if luatex or xetex
  \usepackage{unicode-math} % this also loads fontspec
  \defaultfontfeatures{Scale=MatchLowercase}
  \defaultfontfeatures[\rmfamily]{Ligatures=TeX,Scale=1}
\fi
\usepackage{lmodern}
\ifPDFTeX\else
  % xetex/luatex font selection
\fi
% Use upquote if available, for straight quotes in verbatim environments
\IfFileExists{upquote.sty}{\usepackage{upquote}}{}
\IfFileExists{microtype.sty}{% use microtype if available
  \usepackage[]{microtype}
  \UseMicrotypeSet[protrusion]{basicmath} % disable protrusion for tt fonts
}{}
\makeatletter
\@ifundefined{KOMAClassName}{% if non-KOMA class
  \IfFileExists{parskip.sty}{%
    \usepackage{parskip}
  }{% else
    \setlength{\parindent}{0pt}
    \setlength{\parskip}{6pt plus 2pt minus 1pt}}
}{% if KOMA class
  \KOMAoptions{parskip=half}}
\makeatother
\usepackage{graphicx}
\makeatletter
\newsavebox\pandoc@box
\newcommand*\pandocbounded[1]{% scales image to fit in text height/width
  \sbox\pandoc@box{#1}%
  \Gscale@div\@tempa{\textheight}{\dimexpr\ht\pandoc@box+\dp\pandoc@box\relax}%
  \Gscale@div\@tempb{\linewidth}{\wd\pandoc@box}%
  \ifdim\@tempb\p@<\@tempa\p@\let\@tempa\@tempb\fi% select the smaller of both
  \ifdim\@tempa\p@<\p@\scalebox{\@tempa}{\usebox\pandoc@box}%
  \else\usebox{\pandoc@box}%
  \fi%
}
% Set default figure placement to htbp
\def\fps@figure{htbp}
\makeatother
\setlength{\emergencystretch}{3em} % prevent overfull lines
\providecommand{\tightlist}{%
  \setlength{\itemsep}{0pt}\setlength{\parskip}{0pt}}
\usepackage{amsmath, amssymb}
\usepackage{booktabs}
\usepackage{caption}
\usepackage{float}
\usepackage{geometry}
\usepackage{lmodern}
\usepackage{placeins}
\usepackage{setspace}
\usepackage{threeparttable}
\usepackage{fancyhdr}
\pagestyle{fancy}
\fancyhf{} % clear default header/footer
\cfoot{\thepage} % page number on bottom-right
\renewcommand{\headrulewidth}{0pt}
\usepackage{bookmark}
\IfFileExists{xurl.sty}{\usepackage{xurl}}{} % add URL line breaks if available
\urlstyle{same}
\hypersetup{
  pdftitle={Does Foreign Aid Improve Educational Outcomes? Evidence from Uganda},
  pdfauthor={Riana Ramonjamanana; Charlene R. Ramos},
  hidelinks,
  pdfcreator={LaTeX via pandoc}}

\title{Does Foreign Aid Improve Educational Outcomes? Evidence from
Uganda}
\author{Riana Ramonjamanana \and Charlene R. Ramos}
\date{2025-11-12}

\begin{document}
\maketitle




\section{Introduction}
\section{Data}

\begin{table}[!htbp]
\centering
\caption{Summary Statistics by Year}
\begin{threeparttable}
\begin{tabular}{lcccc}
\toprule
\multicolumn{1}{l}{} & \textbf{2014} & \textbf{2015} & \textbf{2016} & \textbf{2017} \\
\midrule
\multicolumn{5}{l}{\textbf{Education Outcomes}} \\
\quad Primary gross enrollment rate & 125.09 & 111.10 & 128.81 & 115.44 \\
& (43.34) & (26.91) & (66.62) & (30.33) \\~\\
\quad Secondary gross enrollment rate & 29.62 & 23.61 & 28.64 & 24.29 \\
& (15.89) & (13.49) & (19.20) & (12.41)\\
\midrule
\quad N & 123 & 123 & 123 & 123 \\
\midrule
\addlinespace
\multicolumn{5}{l}{\textbf{International Aid}} \\
\quad Total transaction amount & NA
& 221,755 
& 223,778 
& 162,984\\

& 
& (49,162) 
& (17,249) 
& (79,449)\\~\\

\quad Even-split disbursement & NA 
& 179,222 
& 178,038 
& 86,468\\

&  
& (83,186) 
& (144,796) 
& (45,896)\\~\\

\quad Total disbursement & NA
& 681,808 
& 434,891 
& 293,221 \\

& 
& (115,395) 
& (207,273) 
& (119,302)\\
\midrule
\quad N & 0
& 68 
& 12 
& 21\\
\bottomrule
\end{tabular}
\begin{tablenotes}\small
\item Notes: Values are means with standard deviations in parentheses. Aid data are from Uganda's Aid Management Platform that tracks geocoded projects' disbursements. Even-split disbursement represent evenly allocated total disbursement amounts between project start and end years. Education data are from the Uganda Bureau of Statistics.
\end{tablenotes}
\end{threeparttable}
\end{table}
\FloatBarrier

\section{Methodology}
\section{Results}

\begin{table}[htbp]
\centering
\caption{Estimated Effect of Aid on Educational Outcomes}
\label{tab:aid_edu}
\begin{threeparttable}
\begin{tabular}{lcccccccc}
\toprule
                    & (1) L      & (2) L+C     & (3) L+C+FE    & (4) L+C+FE+YFE
                    & (5) Q      & (6) Q+C     & (7) Q+C+FE    & (8) Q+C+FE+YFE \\
\midrule
$Aid_{i,t-1}$        & a1         & a2          & a3            & a4
                     & b1         & b2          & b3            & b4 \\
                     & (se1)      & (se2)       & (se3)         & (se4)
                     & (se5)      & (se6)       & (se7)         & (se8) \\
$Aid_{i,t-1}^2$      &            &             &               & 
                     & c1         & c2          & c3            & c4 \\
                     &            &             &               & 
                     & (se9)      & (se10)      & (se11)        & (se12) \\
\addlinespace
Controls             & No         & Yes         & Yes           & Yes
                     & No         & Yes         & Yes           & Yes \\
FE (district)        & No         & No          & Yes           & Yes
                     & No         & No          & Yes           & Yes \\
FE (year)            & No         & No          & No            & Yes
                     & No         & No          & No            & Yes \\
Squared term incl.?  & No         & No          & No            & No
                     & Yes        & Yes         & Yes           & Yes \\
\midrule
Observations         & …          & …           & …             & …
                     & …          & …           & …             & … \\
$R^{2}$              & …          & …           & …             & …
                     & …          & …           & …             & … \\
\bottomrule
\end{tabular}
\begin{tablenotes}
\footnotesize
Notes: Standard errors (in parentheses) clustered at the district level.
L = linear; Q = quadratic; C = controls; FE = fixed effects; YFE = year fixed effects.
\end{tablenotes}
\end{threeparttable}
\end{table}

\FloatBarrier

\section{Discussion}
\section{Conclusion}




\end{document}
